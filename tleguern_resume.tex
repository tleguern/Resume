%%%%%%%%%%%%%%%%%%%%%%%%%%%%%%%%%%%%%%%%%
% Friggeri Resume/CV
% XeLaTeX Template
% Version 1.2 (3/5/15)
%
% This template has been downloaded from:
% http://www.LaTeXTemplates.com
%
% Original authors:
% Adrien Friggeri (adrien@friggeri.net): https://github.com/afriggeri/CV
% Antoine Marandon (antoine@marandon.fr): https://github.com/ntnmrndn/Resume
%
% License:
% CC BY-NC-SA 3.0 (http://creativecommons.org/licenses/by-nc-sa/3.0/)
%
% Important notes:
% This template needs to be compiled with XeLaTeX and the bibliography, if used,
% needs to be compiled with biber rather than bibtex.
%
%%%%%%%%%%%%%%%%%%%%%%%%%%%%%%%%%%%%%%%%%

\documentclass[]{friggeri-cv} % Add 'print' as an option into the square bracket to remove colors from this template for printing


\begin{document}

\header{Tristan}{Le Guern}{Freelance DevOps/SRE}{Gwir ha leal}

%-------------------------------------------------------------------------------
%	SIDEBAR SECTION
%-------------------------------------------------------------------------------

\begin{aside} % In the aside, each new line forces a line break
\section{Contact}
~
\textbf{Location}: France
~
\textbf{Phone}: \href{tel:0033665630252}{\underline{+33 (0)6 65 63 02 52}}
~
\textbf{Email/XMPP}: \href{mailto:tleguern@bouledef.eu}{\underline{tleguern@bouledef.eu}}
\section {Links}
~
Github: \href{https://github.com/tleguern}{@tleguern}
\section{Personal}
~
\textbf{Nationality}
French
\section{Skills}
~
\textbf{Programming Languages:} C, Shell, Python
\textbf{Operating Systems:} OpenBSD, Linux
\textbf{DevOps:} Ansible, Terraform, Packer, cloud-init.
\textbf{Human Languages:} French (Mother Tongue), English (Fluent), Breton (Beginner)
\textbf{Others:} \LaTeX
\section{Hobbies}
~
Brewing, wine making, cheese making, DIY
\end{aside}

%-------------------------------------------------------------------------------
%	WORK EXPERIENCE SECTION
%-------------------------------------------------------------------------------

\section{Experience}

\subsection{Gwern}
\begin{entrylist}
\entry {2020--2022} {Remote Senior DevOps for Deveryware} {Brest, Brittany} {
  \emph{Resident Ansible expert}
  \begin{itemize}
    \item Trained and mentored Junior DevOps employees.
    \item Teached safe shell script development.
    \item Maintenance of many public and private Ansible roles.
    \item Push for the usage of private Ansible collections.
  \end{itemize}
}
\entry {2019--2020} {Remote DevOps for Deveryware} {Brest, Brittany} {
  \emph{Responsible for the improvements of the developer tools.}
  \begin{itemize}
    \item Improved the installation and upgrade of SonarQube, Redmine and Wekan through dedicated Ansible roles.
    \item Managed an internal mirror of external git repositories (private Ansible Galaxy).
    \item Rewrote shell scripts and Ansible roles for efficiency.
    \item Contributed to open source projects (Ansible's community.general collection)
    \item Deployment of security tools Tenable and Nessus.
  \end{itemize}
}
\entry {2018--2019} {Remote DevOps for Deveryware} {Limerick, Ireland} {
  \emph{Responsible for the deployment of a mobile app's backend.}
  \begin{itemize}
    \item Handled 240 servers split between five environments.
    \item Overhauled and cleaned up many Ansible roles and playbooks, amounting to roughly 12000 lines of code.
    \item Rewrote most shell scripts to improve their safety and performance.
    \item Contribution to open source projects (ansible-haproxy, ansible-wazuh-agent, ansible-rcron, ansible-matomo).
    \item Deployment of security tool Wazuh.
    \item Security tests on the VRRP protocol.
    \item Handled images creation with Packer and deployment with Terraform.
    \item Became maintainer of public role Deveryware/ansible-haproxy.
    \item Wrote extensive documentation dedicated to oncall ops.
    \item Trained an internal DevOps on the job.
  \end{itemize}
}
\entry {2017--2018} {Remote DevOps for Ahrefs} {Paris, France} {
  \emph{Worked in an Ops team}
  \begin{itemize}
    \item Managed multiple Elasticsearch clusters reaching thousands of nodes.
    \item Configuration management with Puppet.
  \end{itemize}
}
\end{entrylist}

\pagebreak

\subsection{Orange}
\begin{entrylist}
\entry {2014--2017} {Sysadmin/DevOps} {Paris, France} {
  \emph{Full time in the subsidiary hosting services for the rest of the company.}
  \begin{itemize}
    \item Administered a Xymon based monitoring solution, handling more than 5000 ressources (servers, services, applications, URL...) and 500 agents.
    \item Managed internal Xymon fork and contributions to upstream.
    \item Developped custom monitoring agents (Perl, Python, shell).
    \item Wrote the documentation for each and every tool.
    \item Wrote the disaster recovery procedure.
    \item Fixed Ansible playbooks to ensure idempotence during deployments.
    \item Implemented dynamic DNS updates in the internal DNS API.
    \item Various improvements to the DNS management tools.
    \item Contributed to nsdiff (https://github.com/fanf2/nsdiff).
  \end{itemize}
}
\end{entrylist}

%------------------------------------------------

% \subsection{Mind Technologies}
% \begin{entrylist}
% \entry {2014--2014} {Network engineer} {Paris, France} {
%   \emph{Fulltime in a company offering expertise in managed network infrastructures.}
% }
% \end{entrylist}

%------------------------------------------------

%\subsection{Linagora}
%\begin{entrylist}
%\entry {2012--2014} {System engineer} {Paris, France} {
%  \begin{itemize}
%    \item Administered a Xen cluster and helped clients into building their hosted services.
%    \item Migrated the old clients ressources from dedicated or shared hostings to virtual machines with Puppet as a deployment tool.
%    \item Implemented p2v (physical to virtual) migration solutions.
%    \item Developped an internal domain name monitoring solution.
%    \item Administered multiple Nagios based monitoring solutions for various clients.
%    \item Handled a three months classified mission for the French military involving documentation, training and monitoring.
%    \item Replaced a team of four sysadmins at Cergy-Pontoise University, worked on the complete overhaul of the SMTP, LDAP and DNS services.
%  \end{itemize}
%}
%\end{entrylist}

%------------------------------------------------

%\subsection{Lab'Free}
%\begin{entrylist}
%\entry {2010--2011} {C++ Developer} {Paris, France} {
%  \emph{Internship. Developed and oversaw implementation of a video streaming software, in C++ and JavaScript.}
%}
%\end{entrylist}

%-------------------------------------------------------------------------------
%	EDUCATION SECTION
%-------------------------------------------------------------------------------

\section{Education}

\begin{entrylist}
\entry
{2008--2013}
{Master's degree {\normalfont of Computer Sciences at EPITECH}}
{Paris, France}
{\emph{5 year course}}
\entry
{2011--2012}
{Exchange Student {\normalfont at Keele University}}
{Newcastle-under-Lyme, England}
{\emph{Computer Sciences course}}
\entry
{2008}
{Baccalauréat {\normalfont at Lycée Colbert}}
{Lorient, France}
{}
\end{entrylist}

%-------------------------------------------------------------------------------
%	OPEN SOURCE
%-------------------------------------------------------------------------------

\section{Open Source}
\begin{entrylist}
\entry
{2020}
{Ansible modules for Proxmox VE}
{Maintainer}
{Maintainer for Ansible's modules proxmox, proxmox\_kvm and proxmox\_template inside collection community.general}
\entry
{2018}
{Various Ansible roles}
{Developer}
{Wrote various ansible modules for matomo, keepalived, haproxy, ...}
\entry
{2018}
{www.libravatar.org}
{Q/A}
{Team member dedicated to Q/A.}
\entry
{2018}
{libravatar.cgi}
{Developer}
{Small, secure implementation of the Libravatar protocol for OpenBSD}
%\entry
%{2011}
%{libtuntap}
%{Developer}
%{Popular portable library for the creation and management of virtual network interfaces.}
%\entry
%{2011--2013}
%{tNETacle (decentralized VPN)}
%{Developer}
%{C/C++ solution developed with a 8 students team \\
%\emph{Asymmetric cryptography, UPNP, IPC, UDP, TCP, TLS, DTLS, portability, compression...}}
\end{entrylist}

\end{document}
